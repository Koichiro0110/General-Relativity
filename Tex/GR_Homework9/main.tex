\documentclass[12pt]{article}
\usepackage{fullpage}
\usepackage{graphicx}
\usepackage{hyperref}
\usepackage{bm}
\usepackage{amsmath}
\usepackage{amssymb}
\usepackage{derivative}
\usepackage{bm}
\usepackage{comment}
\usepackage{cancel}
\usepackage{xcolor}
\usepackage{float}

\renewcommand{\arraystretch}{1.5}

\begin{document}
\title{General Relativity: Homework 9}
\author{Koichiro Takahashi}
\maketitle

\section*{Problem.1}
The Kerr metric gives the line element
\begin{align}
ds^2 = - \left(1 - \frac{2 G M r}{\rho^2} \right) dt^2 & - \frac{2 G M a r \sin^2{\theta}}{\rho^2} \left(dt d\phi + d\phi dt\right) \nonumber \\
&+ \frac{\rho^2}{\Delta} dr^2 + \rho^2 d\theta^2 + \frac{\sin{\theta}}{\rho^2} \left[ \left(r^2 + a^2\right)^2 - a^2 \Delta \sin^2{\theta} \right] d\phi^2
\end{align}
where
\begin{align}
\Delta = r^2 - 2 G M r + a^2, \quad \rho^2 = r^2 + a^2 \cos^2{\theta}, \quad - \frac{G M}{c} \leq a \leq \frac{G M}{c}
\end{align}
Here, we assume $\theta = \pi/2, a = G M$, therefore,
\begin{align}
ds^2 = - \left(1 - \frac{2 G M r}{\rho^2} \right) dt^2 & - \frac{2 G^2 M^2 r}{\rho^2} \left(dt d\phi + d\phi dt\right) \nonumber \\
&+ \frac{\rho^2}{\Delta} dr^2 + \rho^2 d\theta^2 + \frac{1}{\rho^2} \left[ \left(r^2 + G^2M^2\right)^2 - G^2M^2 \Delta \right] d\phi^2
\end{align}
where
\begin{align}
\Delta = r^2 - 2 G M r + G^2 M^2 = \left(r - GM\right)^2, \quad \rho^2 = r^2
\end{align}
By substituting the parameters above,
\begin{align}
ds^2 = - \left(1 - \frac{2 G M}{r} \right) dt^2 & - \frac{2 G^2 M^2}{r} \left(dt d\phi + d\phi dt\right) \nonumber \\
&+ \frac{r^2}{\left(r - GM\right)^2} dr^2 + r^2 d\theta^2 + \frac{1}{r^2} \left[ \left(r^2 + G^2M^2\right)^2 - G^2M^2 \left(r - GM\right)^2 \right] d\phi^2
\end{align}
(i) For the vector to be spacelike, it has to satisfy
\begin{align}
g_{\mu \nu} V^{\mu} V^{\nu} > 0
\end{align}
For $GM < r < 2GM$, when $V^{\mu} = \left(1, 0, 0, 0\right)$,
\begin{align}
g_{\mu \nu} V^{\mu} V^{\nu} = g_{tt} = - \left(1 - \frac{2 G M}{r} \right) > 0 ~~\left(\because GM < r\right)
\end{align}
When $V^{\mu} = \left(0, 1, 0, 0\right)$,
\begin{align}
g_{\mu \nu} V^{\mu} V^{\nu} = g_{rr} = \frac{r^2}{\left(r - GM\right)^2} > 0 
\end{align}
When $V^{\mu} = \left(0, 0, 1, 0\right)$,
\begin{align}
g_{\mu \nu} V^{\mu} V^{\nu} = g_{\theta \theta} = r^2 > 0 
\end{align}
When $V^{\mu} = \left(0, 0, 0, 1\right)$,
\begin{align}
g_{\mu \nu} V^{\mu} V^{\nu} &= g_{\phi \phi} = \frac{1}{r^2} \left[ \left(r^2 + G^2M^2\right)^2 - G^2M^2 \left(r - GM\right)^2 \right] \nonumber \\
&= \frac{1}{r^2} \left[ r^4 + 2G^2M^2r^2 + G^4M^4  - G^2M^2 r^2 + 2 G^3M^3 r - G^4M^4 \right] \nonumber \\
&= \frac{1}{r^2} \left[ r^4 + G^2M^2r^2 + 2 G^3M^3 r \right] > 0 
\end{align}
Therefore, all the vectors are spacelike.\\
(ii) Here, the vector $V^{\mu} = (1,0,0,A)$. For the vector to be timelike, it has to satisfy
\begin{align}
g_{\mu \nu} V^{\mu} V^{\nu} < 0
\end{align}
$\Leftrightarrow$
\begin{align}
- \left(1 - \frac{2 G M}{r} \right) - 2 A \cdot \frac{2 G^2 M^2}{r} + A^2 \cdot \left( r^2 + G^2M^2 + \frac{2 G^3M^3}{r} \right)  < 0
\end{align}
$\Leftrightarrow$
\begin{align}
\left(2GM - r\right) - 2A \cdot 2 G^2 M^2 + A^2 \cdot \left( r^3 + G^2 M^2r + 2 G^3 M^3 \right)  < 0
\end{align}
At $r = 3GM/2$,
\begin{align}
\frac{1}{2} - 2A \cdot 2GM + A^2 \cdot G^2M^2\left( \frac{27}{8} + \frac{3}{2} + 2\right)  < 0
\end{align}
$\Leftrightarrow$
\begin{align}
1 - 2A \cdot 4GM + A^2 \cdot \frac{55}{4}G^2M^2  < 0
\end{align}
$\Leftrightarrow$
\begin{align}
4 - 2A \cdot 16GM + A^2 \cdot 55G^2M^2  < 0
\end{align}
Thus, if we solve for the equality,
\begin{align}
A = \frac{16 GM \pm 6GM}{55 G^2M^2} = \frac{2}{11GM}, \frac{32}{55GM}
\end{align}
Therefore, for $A$ to be timelike,
\begin{align}
\frac{2}{11GM} < A < \frac{32}{55GM}
\end{align}
(iii) A ZAMO at $\theta = \pi/2, p^{r} = p^{\theta} = 0$, here,
\begin{align}
p^{\phi} = p_{t} g^{t \phi} = - p_{t} \frac{2 G^2 M^2}{r} \Leftrightarrow \frac{p^{\phi}}{p_{t}} = - \frac{2 G^2 M^2}{r}
\end{align}
\begin{align}
p^{t} = p_{t} g^{t t} + p_{\phi} g^{\phi t} = - \frac{p_{t}}{\left(1 - \frac{2 G M}{r} \right)} \Leftrightarrow p_{t} = - \left(1 - \frac{2 G M}{r} \right) p^{t}
\end{align}
Thus,
\begin{align}
\frac{p^{\phi}}{p^{t}} = \frac{2 G^2 M^2}{\left(r - 2GM \right)}
\end{align}
Therefore,
\begin{align}
\frac{d \phi}{d t} = \frac{d \phi}{d \tau} \frac{d \tau}{dt} = \frac{p^{\phi}}{p^{t}} = \frac{2 G^2 M^2}{\left(r - 2GM \right)}
\end{align}
\section*{Problem.2}
The line element is given by
\begin{align}
ds^2 = - \left(1 - \frac{2}{r} \right) dv^2 & + 2 dv dr + r^2 d\theta^2 + r^2 \sin^2{\theta} d\phi^2
\end{align}
in Eddington-Finkelstein coordinates.\\
(i) Here, $u^{\theta} = u^{\phi} = 0$. The acceleration of the massive observer is given by the geodesic equation
\begin{align}
\frac{d^2 v}{d \tau^2} &= \frac{d u^{v}}{d \tau} = - \Gamma^{v}_{\mu \nu} u^{\mu} u^{\nu} = - \Gamma^{v}_{vv} u^{v} u^{v} - \Gamma^{v}_{r r} u^{r} u^{r} - \Gamma^{v}_{v r} u^{v} u^{r} - \Gamma^{v}_{v r} u^{r} u^{v} \nonumber \\
&= - \Gamma^{v}_{vv} u^{v} u^{v} - \Gamma^{v}_{r r} u^{r} u^{r} - 2 \Gamma^{v}_{v r} u^{v} u^{r} \left(\because \Gamma^{v}_{v r} = \Gamma^{v}_{r v}\right)
\end{align}
Similarly,
\begin{align}
\frac{d^2 r}{d \tau^2} &= \frac{d u^{r}}{d \tau} = - \Gamma^{r}_{vv} u^{v} u^{v} - \Gamma^{r}_{r r} u^{r} u^{r} - 2 \Gamma^{r}_{v r} u^{v} u^{r} \left(\because \Gamma^{r}_{v r} = \Gamma^{r}_{r v}\right)
\end{align}
The Christoffel symbols are given by
\begin{align}
\Gamma^{\nu}_{\alpha \beta} = \frac{1}{2} g^{\nu \sigma} \left(\partial_{\alpha} g_{\beta \sigma} + \partial_{\beta} g_{\alpha \sigma} - \partial_{\sigma} g_{\alpha \beta}\right)
\end{align}
Therefore,
\begin{align}
\Gamma^{v}_{vv} &= \frac{1}{2} g^{v \sigma} \left(\partial_{v} g_{v \sigma} + \partial_{v} g_{v \sigma} - \partial_{\sigma} g_{vv}\right)\\
&= - \frac{1}{2} g^{v r} \partial_{r} g_{vv} ~~\left(\because \partial_{v} g_{\alpha \beta} = 0 \right) \\
&= \frac{1}{2} \partial_{r} \left(1 - \frac{2}{r} \right) = \frac{1}{r^2}
\end{align}
\begin{align}
\Gamma^{v}_{rr} = \frac{1}{2} g^{v \sigma} \left(\partial_{r} g_{r \sigma} + \partial_{r} g_{r \sigma} - \partial_{\sigma} g_{r r}\right) 
= 0
\end{align}
\begin{align}
\Gamma^{v}_{v r} &= \frac{1}{2} g^{v \sigma} \left(\partial_{v} g_{r \sigma} + \partial_{r} g_{v \sigma} - \partial_{\sigma} g_{v r}\right) = \frac{1}{2} g^{v v} \partial_{r} g_{v v} = 0 ~~\left(\because g_{v v} = 0 \right)
\end{align}
\begin{align}
\Gamma^{r}_{vv} &= \frac{1}{2} g^{r \sigma} \left(\partial_{v} g_{v \sigma} + \partial_{v} g_{v \sigma} - \partial_{\sigma} g_{vv}\right)\\
&= - \frac{1}{2} g^{r r} \partial_{r} g_{vv} = \frac{1}{2} g_{vv} \partial_{r} g_{vv}  = \frac{1}{2} \left(1 - \frac{2}{r} \right) \partial_{r} \left(1 - \frac{2}{r} \right) = \frac{1}{r^2} \left(1 - \frac{2}{r} \right)
\end{align}
\begin{align}
\Gamma^{r}_{rr} = \frac{1}{2} g^{r \sigma} \left(\partial_{r} g_{r \sigma} + \partial_{r} g_{r \sigma} - \partial_{\sigma} g_{r r}\right) = 0
\end{align}
\begin{align}
\Gamma^{r}_{vr} &= \frac{1}{2} g^{r \sigma} \left(\partial_{v} g_{r \sigma} + \partial_{r} g_{v \sigma} - \partial_{\sigma} g_{v r}\right)
= \frac{1}{2} g^{r v} \partial_{r} g_{v v} = - \frac{1}{r^2}
\end{align}
Thus, we get
\begin{align}
\frac{d^2 v}{d \tau^2} &= - \frac{\left(u^{v}\right)^2}{r^2}\\
\frac{d^2 r}{d \tau^2} &= - \frac{\left(u^{v}\right)^2}{r^2} \left(1 - \frac{2}{r} \right) + \frac{2u^{v} u^{r}}{r^2} 
\end{align}
After we solve this ODEs numerically, we get $\tau_{c} = 20$. The code is attached at the end.\\
(iii)
\begin{align}
\tau_{c} \frac{GM}{c}  &= 73.33 ~\mathrm{years} \approx 2\cdot 10^9 ~\mathrm{s}\\
&\Leftrightarrow M \approx \frac{2\cdot 10^9\cdot c}{\tau_{c} G} = \frac{2\cdot 10^9\cdot 3 \cdot 10^8}{20 \cdot 6.67 \cdot 10^{-11}} \approx 
4.5 \cdot 10^{26}~\mathrm{kg}
\end{align}


\end{document}