\documentclass[12pt]{article}
\usepackage{fullpage}
\usepackage{graphicx}
\usepackage{hyperref}
\usepackage{bm}
\usepackage{amsmath}
\usepackage{amssymb}
\usepackage{derivative}
\usepackage{bm}
\usepackage{comment}
\usepackage{cancel}
\usepackage{xcolor}
\usepackage{float}

\renewcommand{\arraystretch}{1.5}

\begin{document}
\title{General Relativity: Homework 10}
\author{Koichiro Takahashi}
\maketitle

\section*{Problem.1}
Here, the linearized gravity
\begin{align}
g_{\mu \nu} = \eta_{\mu \nu} + h_{\mu \nu}
\end{align}
where $\eta_{\mu \nu}$ is the Minkowski metric and
\begin{align}
h_{\mu \nu} = \epsilon
\begin{pmatrix}
- \cos\left(t - y\right) & 0 & 2 \cos\left(t - y\right) & 2 \sin\left(t - y\right)\\
0 & - \cos\left(t - y\right) & 0 & 0 \\
2 \cos\left(t - y\right) & 0 & -3 \cos\left(t - y\right) & -2 \sin\left(t - y\right) \\
2 \sin\left(t - y\right) & 0 & -2 \sin\left(t - y\right) & \cos\left(t - y\right)
\end{pmatrix}
\end{align}
where $\epsilon \ll 1$.\\
Under a coordinate transformation
\begin{align}
x^{\mu'} = x^{\mu} + \zeta^{\mu}
\end{align}
the metric perturbation transform as
\begin{align}
h'_{\mu \nu} = h_{\mu \nu} - \partial_{\mu} \zeta_{\nu} - \partial_{\nu} \zeta_{\mu}
\end{align}
Clearly, $h'_{\mu \nu} = h'_{\nu \mu}$ due to $h_{\mu \nu} = h_{\nu \mu}$. In principle, $\zeta_{\nu} = \zeta_{\nu}(t,x,y,z).$ To keep $h'_{xx}, h'_{xz}, h'_{zz}$ unchanged, we need
\begin{align}
h'_{xx}= h_{xx} = h_{xx} - 2 \partial_{x} \zeta_{x} = 0 &\Leftrightarrow \partial_{x} \zeta_{x} = 0 \\
h'_{xz} = h'_{zx} = h_{xz} = h_{xz} - \partial_{x} \zeta_{z} - \partial_{z} \zeta_{x} &\Leftrightarrow  \partial_{z} \zeta_{x} = - \partial_{x} \zeta_{z} \\
h'_{zz} = h_{zz} = h_{zz} - 2 \partial_{z} \zeta_{z} &\Leftrightarrow \partial_{z} \zeta_{z} = 0
\end{align}
Also, all other components of $h'_{\mu \nu}$ to zero, so that
\begin{align}
0 = h'_{t t} = h_{tt} - 2 \partial_{t} \zeta_{t} &\Leftrightarrow \partial_{t} \zeta_{t} = - \frac{\epsilon}{2} \cos\left(t - y\right) \\
0 = h'_{y y} = h_{yy} - 2 \partial_{y} \zeta_{y} &\Leftrightarrow \partial_{y} \zeta_{y} = - \frac{3 \epsilon}{2} \cos\left(t - y\right)
\end{align}
Therefore, we can choose them as independent of $x, z$,
\begin{align}
\zeta_{t} &= - \frac{\epsilon}{2} \sin\left(t - y\right) \\
\zeta_{y} &= \frac{3 \epsilon}{2} \sin\left(t - y\right)
\end{align}
Then,
\begin{align}
h'_{t y} = h'_{y t} &= h_{ty} - \partial_{t} \zeta_{y} - \partial_{y} \zeta_{t} = 2 \epsilon \cos\left(t - y\right) - \frac{3 \epsilon}{2} \cos\left(t - y\right) - \frac{\epsilon}{2} \cos\left(t - y\right) = 0
\end{align}
Now, to determine $\zeta_{x}, \zeta_{z}$,
\begin{align}
0 = h'_{t x} = h'_{x t} &= h_{tx} - \partial_{t} \zeta_{x} - \partial_{x} \zeta_{t} = - \partial_{t} \zeta_{x} \\
0 = h'_{y x} = h'_{x y} &= h_{yx} - \partial_{y} \zeta_{x} - \partial_{x} \zeta_{y} = - \partial_{y} \zeta_{x} \\
0 = h'_{t z} = h'_{z t} &= h_{tz} - \partial_{t} \zeta_{z} - \partial_{z} \zeta_{t} = 2 \epsilon \sin\left(t - y\right) - \partial_{t} \zeta_{z} \\
0 = h'_{y z} = h'_{z y} &= h_{yz} - \partial_{y} \zeta_{z} - \partial_{z} \zeta_{y} = - 2 \epsilon \sin\left(t - y\right) - \partial_{y} \zeta_{z}
\end{align}
Therefore, we can choose them as
\begin{align}
\zeta_{x} &= 0\\
\zeta_{z} &= - 2 \epsilon \cos\left(t - y\right)
\end{align}
Then,
\begin{align}
h'_{x z} = h'_{z x} &= h_{xz} - \partial_{x} \zeta_{z} - \partial_{z} \zeta_{x} = 0
\end{align}
Therefore,
\begin{align}
\zeta_{t} &= - \frac{\epsilon}{2} \sin\left(t - y\right) \\
\zeta_{x} &= 0\\
\zeta_{y} &= \frac{3 \epsilon}{2} \sin\left(t - y\right) \\
\zeta_{z} &= - 2 \epsilon \cos\left(t - y\right)
\end{align}

\section*{Problem.2}
The quadrupole moment of the rod is given by
\begin{align}
I_{i j} &= \frac{M L^2}{12}
\begin{pmatrix}
1 & 0 & 0\\
0 & \cos^2\left(\Omega t\right) & \sin\left(\Omega t\right) \cos\left(\Omega t\right) \\
0 & \sin\left(\Omega t\right) \cos\left(\Omega t\right) & \sin^2\left(\Omega t\right)
\end{pmatrix}
\\
&= \frac{M L^2}{24}
\begin{pmatrix}
1 & 0 & 0\\
0 & 1 + \cos\left(2 \Omega t\right) & \sin\left(2 \Omega t\right)\\
0 & \sin\left(2 \Omega t\right) & 1 - \cos\left(2 \Omega t\right)
\end{pmatrix}
\\
\end{align}
(i) Here, by the quadrupole formula
\begin{align}
\bar{h}_{i j}(t, \bar{x}) &= \frac{2 G}{D} \frac{d^2}{dt^2} I_{i j}(t - D) \\
&= \frac{G M L^2}{12 D}
\frac{d^2}{dt^2}
\begin{pmatrix}
1 & 0 & 0\\
0 & 1 + \cos\left(2 \Omega t\right) & \sin\left(2 \Omega t\right)\\
0 & \sin\left(2 \Omega t\right) & 1 - \cos\left(2 \Omega t\right)
\end{pmatrix}
\\
&= \frac{\Omega^2 G M L^2}{3 D}
\begin{pmatrix}
0 & 0 & 0\\
0 & - \cos\left(2 \Omega (t-D)\right) & - \sin\left(2 \Omega (t-D)\right)\\
0 & - \sin\left(2 \Omega (t-D)\right) &  \cos\left(2 \Omega (t-D)\right)
\end{pmatrix}
\end{align}
(ii) Here, the observer is on the $y$-axis, so that $D = y$, 
\begin{align}
\bar{h}_{i j}(t, \bar{x}) &= \frac{\Omega^2 G M L^2}{3 y}
\begin{pmatrix}
0 & 0 & 0\\
0 & - \cos\left(2 \Omega (t-y)\right) & - \sin\left(2 \Omega (t-y)\right)\\
0 & - \sin\left(2 \Omega (t-y)\right) &  \cos\left(2 \Omega (t-y)\right)
\end{pmatrix}
\end{align}
Up to the order of $\epsilon$, $h^{\mu \nu} = \eta^{\mu \alpha} \eta^{\nu \beta} h_{\alpha \beta}$. Thus,
\begin{align}
h^{tt} = h_{tt}, \quad h^{ij} = h_{ij}, \quad h^{ti} = - h_{ti}
\end{align}
so that
\begin{align}
\bar{h}^{\mu \nu}(t, \bar{x}) &=
\begin{pmatrix}
\bar{h}^{tt} & \bar{h}^{tx} & \bar{h}^{ty} & \bar{h}^{tz}\\
\bar{h}^{xt} & 0 & 0 & 0\\
\bar{h}^{yt} & 0 & - \frac{\Omega^2 G M L^2}{3 y} \cos\left(2 \Omega (t-y)\right) & - \frac{\Omega^2 G M L^2}{3 y} \sin\left(2 \Omega (t-y)\right)\\
\bar{h}^{zt} & 0 & - \frac{\Omega^2 G M L^2}{3 y} \sin\left(2 \Omega (t-y)\right) &  \frac{\Omega^2 G M L^2}{3 y} \cos\left(2 \Omega (t-y)\right)
\end{pmatrix}
\end{align}
Using the Lorentz gauge,
\begin{align}
\partial_{\mu} \bar{h}^{\mu \nu} = 0
\end{align}
Note that $\bar{h}^{i j}$ is independent of $x, z$, also we ignore $y$-dependence in the amplitude. Therefore,
\begin{align}
\partial_{t} \bar{h}^{tx} + \partial_{y} \bar{h}^{yx} = \partial_{t} \bar{h}^{tx} = 0
\end{align}
Thus, we can choose
\begin{align}
\bar{h}^{tx} = 0
\end{align}
Also,
\begin{align}
\partial_{t} \bar{h}^{ty} + \partial_{y} \bar{h}^{yy} &= \partial_{t} \bar{h}^{ty} - \frac{2 \Omega^3 G M L^2}{3 y} \sin\left(2 \Omega (t-y)\right) = 0\\
& \Rightarrow \partial_{t} \bar{h}^{ty} = \frac{2 \Omega^3 G M L^2}{3 y} \sin\left(2 \Omega (t-y)\right)
\end{align}
Thus, we can choose
\begin{align}
\bar{h}^{ty} = - \frac{\Omega^2 G M L^2}{3 y} \cos\left(2 \Omega (t-y)\right)
\end{align}
Similarly,
\begin{align}
\partial_{t} \bar{h}^{tz} + \partial_{y} \bar{h}^{yz} &= \partial_{t} \bar{h}^{tz} + \frac{2 \Omega^3 G M L^2}{3 y} \cos\left(2 \Omega (t-y)\right) = 0\\
& \Rightarrow \partial_{t} \bar{h}^{tz} = - \frac{2 \Omega^3 G M L^2}{3 y} \cos\left(2 \Omega (t-y)\right)
\end{align}
Thus, we can choose
\begin{align}
\bar{h}^{tz} = - \frac{\Omega^2 G M L^2}{3 y} \sin\left(2 \Omega (t-y)\right)
\end{align}
Now, $\bar{h}^{t i}$ is also independent of $x, z$, also we ignore $y$-dependence in the amplitude. Hence, using $\bar{h}^{t i} = \bar{h}^{i t}$
\begin{align}
\partial_{t} \bar{h}^{tt} + \partial_{y} \bar{h}^{yt} &= \partial_{t} \bar{h}^{tt} + \partial_{y} \bar{h}^{ty} = \partial_{t} \bar{h}^{tt} - \frac{2 \Omega^3 G M L^2}{3 y} \sin\left(2 \Omega (t-y)\right) = 0\\
& \Rightarrow \partial_{t} \bar{h}^{tt} = + \frac{2 \Omega^3 G M L^2}{3 y} \sin\left(2 \Omega (t-y)\right)
\end{align}
Thus, we can choose
\begin{align}
\bar{h}^{tt} = - \frac{\Omega^2 G M L^2}{3 y} \cos\left(2 \Omega (t-y)\right)
\end{align}
From all of the above,
\begin{align}
\bar{h}^{tt} &= - \frac{\Omega^2 G M L^2}{3 y} \cos\left(2 \Omega (t-y)\right) \\
\bar{h}^{tx} &= 0 \\
\bar{h}^{ty} &= - \frac{\Omega^2 G M L^2}{3 y} \cos\left(2 \Omega (t-y)\right) \\
\bar{h}^{tz} &= - \frac{\Omega^2 G M L^2}{3 y} \sin\left(2 \Omega (t-y)\right)
\end{align}
(iii)
\begin{align}
\bar{h}^{\mu \nu}(t, \bar{x}) &=
\begin{pmatrix}
- \frac{\Omega^2 G M L^2}{3 y} \cos\left(2 \Omega (t-y)\right) & 0 & - \frac{\Omega^2 G M L^2}{3 y} \cos\left(2 \Omega (t-y)\right) & - \frac{\Omega^2 G M L^2}{3 y} \sin\left(2 \Omega (t-y)\right)\\
0 & 0 & 0 & 0\\
- \frac{\Omega^2 G M L^2}{3 y} \cos\left(2 \Omega (t-y)\right) & 0 & - \frac{\Omega^2 G M L^2}{3 y} \cos\left(2 \Omega (t-y)\right) & - \frac{\Omega^2 G M L^2}{3 y} \sin\left(2 \Omega (t-y)\right)\\
- \frac{\Omega^2 G M L^2}{3 y} \sin\left(2 \Omega (t-y)\right) & 0 & - \frac{\Omega^2 G M L^2}{3 y} \sin\left(2 \Omega (t-y)\right) &  \frac{\Omega^2 G M L^2}{3 y} \cos\left(2 \Omega (t-y)\right)
\end{pmatrix}
\\
\bar{h}_{\mu \nu}(t, \bar{x}) &=
\begin{pmatrix}
- \frac{\Omega^2 G M L^2}{3 y} \cos\left(2 \Omega (t-y)\right) & 0 & \frac{\Omega^2 G M L^2}{3 y} \cos\left(2 \Omega (t-y)\right) & \frac{\Omega^2 G M L^2}{3 y} \sin\left(2 \Omega (t-y)\right)\\
0 & 0 & 0 & 0\\
\frac{\Omega^2 G M L^2}{3 y} \cos\left(2 \Omega (t-y)\right) & 0 & - \frac{\Omega^2 G M L^2}{3 y} \cos\left(2 \Omega (t-y)\right) & - \frac{\Omega^2 G M L^2}{3 y} \sin\left(2 \Omega (t-y)\right)\\
\frac{\Omega^2 G M L^2}{3 y} \sin\left(2 \Omega (t-y)\right) & 0 & - \frac{\Omega^2 G M L^2}{3 y} \sin\left(2 \Omega (t-y)\right) &  \frac{\Omega^2 G M L^2}{3 y} \cos\left(2 \Omega (t-y)\right)
\end{pmatrix}
\end{align}
Here,
\begin{align}
\bar{h}_{\mu \nu} &= h_{\mu \nu} - \frac{1}{2} \eta_{\mu \nu} h^{\alpha}_{~\alpha} \\
\bar{h}^{\alpha}_{~\alpha} &= - h^{\alpha}_{~\alpha}
\end{align}
so that
\begin{align}
h_{\mu \nu} &= \bar{h}_{\mu \nu} + \frac{1}{2} \eta_{\mu \nu} \bar{h}^{\alpha}_{~\alpha}
\end{align}
Also,
\begin{align}
\bar{h}^{\mu}_{~\nu} &= \bar{h}^{\mu \alpha} \eta_{\alpha \nu} \\
&=
\begin{pmatrix}
\frac{\Omega^2 G M L^2}{3 y} \cos\left(2 \Omega (t-y)\right) & 0 & - \frac{\Omega^2 G M L^2}{3 y} \cos\left(2 \Omega (t-y)\right) & - \frac{\Omega^2 G M L^2}{3 y} \sin\left(2 \Omega (t-y)\right)\\
0 & 0 & 0 & 0\\
- \frac{\Omega^2 G M L^2}{3 y} \cos\left(2 \Omega (t-y)\right) & 0 & - \frac{\Omega^2 G M L^2}{3 y} \cos\left(2 \Omega (t-y)\right) & - \frac{\Omega^2 G M L^2}{3 y} \sin\left(2 \Omega (t-y)\right)\\
- \frac{\Omega^2 G M L^2}{3 y} \sin\left(2 \Omega (t-y)\right) & 0 & - \frac{\Omega^2 G M L^2}{3 y} \sin\left(2 \Omega (t-y)\right) &  \frac{\Omega^2 G M L^2}{3 y} \cos\left(2 \Omega (t-y)\right)
\end{pmatrix}
\\
\Rightarrow \bar{h}^{\alpha}_{~\alpha} &= \frac{\Omega^2 G M L^2}{3 y} \cos\left(2 \Omega (t-y)\right)
\end{align}
Therefore,
\begin{align}
&h_{\mu \nu} = \bar{h}_{\mu \nu} + \frac{1}{2} \eta_{\mu \nu} \bar{h}^{\alpha}_{~\alpha} \\
&=
\begin{pmatrix}
\frac{\Omega^2 G M L^2}{6 y} \cos\left(2 \Omega (t-y)\right) & 0 & \frac{\Omega^2 G M L^2}{3 y} \cos\left(2 \Omega (t-y)\right) & \frac{\Omega^2 G M L^2}{3 y} \sin\left(2 \Omega (t-y)\right)\\
0 & \frac{\Omega^2 G M L^2}{3 y} \cos\left(2 \Omega (t-y)\right) & 0 & 0\\
\frac{\Omega^2 G M L^2}{3 y} \cos\left(2 \Omega (t-y)\right) & 0 & \frac{\Omega^2 G M L^2}{6 y} \cos\left(2 \Omega (t-y)\right) & - \frac{\Omega^2 G M L^2}{3 y} \sin\left(2 \Omega (t-y)\right)\\
\frac{\Omega^2 G M L^2}{3 y} \sin\left(2 \Omega (t-y)\right) & 0 & - \frac{\Omega^2 G M L^2}{3 y} \sin\left(2 \Omega (t-y)\right) &  \frac{2 \Omega^2 G M L^2}{3 y} \cos\left(2 \Omega (t-y)\right)
\end{pmatrix}
\end{align}

\section*{Problem.3}
\begin{align}
ds^2 = \frac{\alpha^2}{\cos^2{\chi}} \left( - dt^2 + d \chi^2 \right)
\end{align}
so that the metric is given by
\begin{align}
g_{\mu \nu} =
\begin{pmatrix}
- \frac{\alpha^2}{\cos^2{\chi}} & 0 \\
0 & \frac{\alpha^2}{\cos^2{\chi}}
\end{pmatrix}
\end{align}
(i)
\begin{align}
\partial_{\chi} g_{\mu \nu} =
\begin{pmatrix}
- \frac{2 \alpha^2 \sin{\chi}}{\cos^3{\chi}} & 0 \\
0 & \frac{2 \alpha^2 \sin{\chi}}{\cos^3{\chi}}
\end{pmatrix}
\end{align}
The Christoffel symbols are given by
\begin{align}
\Gamma^{\alpha}_{\mu \nu} = \frac{1}{2} g^{\alpha \sigma} \left(\partial_{\mu} g_{\nu \sigma} + \partial_{\nu} g_{\mu \sigma} - \partial_{\sigma} g_{\mu \nu} \right)
\end{align}
Particularly, for the diagonal metric,
\begin{align}
\Gamma^{\alpha}_{\mu \nu} = \frac{1}{2} g^{\alpha \alpha} \left(\partial_{\mu} g_{\nu \alpha} + \partial_{\nu} g_{\mu \alpha} - \partial_{\alpha} g_{\mu \nu} \right)
\end{align}
$\Rightarrow$
\begin{align}
\Gamma^{t}_{\mu \nu} &= \frac{1}{2} g^{tt} \left(\partial_{\mu} g_{\nu t} + \partial_{\nu} g_{\mu t} - \partial_{t} g_{\mu \nu} \right) \\
&= \frac{1}{2} g^{tt} \left(\partial_{\mu} g_{\nu t} + \partial_{\nu} g_{\mu t} \right) \\
\end{align}
$\Rightarrow$
\begin{align}
\Gamma^{t}_{t \chi} = \Gamma^{t}_{\chi t} &= \frac{1}{2} g^{tt} \partial_{\chi} g_{t t} = \frac{1}{2} \frac{\cos^2{\chi}}{\alpha^2} \frac{2 \alpha^2 \sin{\chi}}{\cos^3{\chi}} = \tan{\chi}\\
\end{align}
and
\begin{align}
\Gamma^{\chi}_{\mu \nu} &= \frac{1}{2} g^{\chi \chi} \left(\partial_{\mu} g_{\nu \chi} + \partial_{\nu} g_{\mu \chi} - \partial_{\chi} g_{\mu \nu} \right)
\end{align}
$\Rightarrow$
\begin{align}
\Gamma^{\chi}_{\chi \chi} &= \frac{1}{2} g^{\chi \chi} \left(\partial_{\chi} g_{\chi \chi} + \partial_{\chi} g_{\chi \chi} - \partial_{\chi} g_{\chi \chi} \right) \\
&= \frac{1}{2} g^{\chi \chi} \partial_{\chi} g_{\chi \chi} = \tan{\chi}
\end{align}
$\Rightarrow$
\begin{align}
\Gamma^{\chi}_{t t} &= \frac{1}{2} g^{\chi \chi} \left(\partial_{t} g_{t \chi} + \partial_{t} g_{t \chi} - \partial_{\chi} g_{t t} \right) \\
&= - \frac{1}{2} g^{\chi \chi} \partial_{\chi} g_{t t} = \tan{\chi}
\end{align}
(ii) The geodesic equation
\begin{align}
\frac{d u^{\alpha}}{d \tau} = - \Gamma^{\alpha}_{\mu \nu} u^{\mu} u^{\nu}
\end{align}
\begin{align}
\frac{d u^{t}}{d \tau} = - \Gamma^{t}_{\mu \nu} u^{\mu} u^{\nu} = - 2 \Gamma^{t}_{t \chi} u^{t} u^{\chi} = - 2 \tan{\chi} \, u^{t} u^{\chi}
\end{align}
\begin{align}
\frac{d u^{\chi}}{d \tau} &= - \Gamma^{\chi}_{\mu \nu} u^{\mu} u^{\nu} = - \Gamma^{\chi}_{\chi \chi} (u^{\chi})^2 - \Gamma^{\chi}_{t t} (u^{t})^2 \\
&= - \tan{\chi} \, \left((u^{\chi})^2 + (u^{t})^2\right)
\end{align}
(iii) Here,
\begin{align}
v^{\chi} = \frac{d \chi}{d t} = \frac{u^{\chi}}{u^{t}} \Leftrightarrow u^{\chi} = u^{t} v^{\chi}
\end{align}
Now,
\begin{align}
\frac{d u^{\chi}}{d \tau} &= \frac{d  \left(u^{t}v^{\chi}\right)}{d \tau} = u^{t} \frac{d  v^{\chi}}{d \tau} +  v^{\chi} \frac{d  u^{t}}{d \tau}
= u^{t} \frac{d  v^{\chi}}{d \tau} - 2 \tan{\chi} \, u^{\chi} u^{t} v^{\chi} \\
&= - \tan{\chi} \, \left((u^{\chi})^2 + (u^{t})^2\right) \\
\end{align}
$\Leftrightarrow$
\begin{align}
u^{t} \frac{d  v^{\chi}}{d \tau} = 2 \tan{\chi} \, u^{\chi} u^{t} v^{\chi} - \tan{\chi} \, \left((u^{\chi})^2 + (u^{t})^2\right)
\end{align}
$\Leftrightarrow$
\begin{align}
(u^{t})^2 \frac{d  v^{\chi}}{d t} = 2 \tan{\chi} \, u^{t} v^{\chi} u^{t} v^{\chi} - \tan{\chi} \, \left((u^{t} v^{\chi})^2 + (u^{t})^2\right)
\end{align}
$\Leftrightarrow$
\begin{align}
\frac{d  v^{\chi}}{d t} = \tan{\chi} \, \left[(v^{\chi})^2 - 1\right]
\end{align}
\end{document}