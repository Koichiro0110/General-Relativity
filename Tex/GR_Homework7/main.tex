\documentclass[12pt]{article}
\usepackage{fullpage}
\usepackage{graphicx}
\usepackage{hyperref}
\usepackage{bm}
\usepackage{amsmath}
\usepackage{amssymb}
\usepackage{derivative}
\usepackage{bm}
\usepackage{comment}
\usepackage{cancel}
\usepackage{xcolor}
\usepackage{float}

\renewcommand{\arraystretch}{1.5}

\begin{document}
\title{General Relativity: Homework 7}
\author{Koichiro Takahashi}
\maketitle

\section*{Problem.1}
(i)
A satellite is orbiting at constant radius $r = 6$ in the equatorial plane $(\theta = \pi/2)$ with the Schwarzschild metric in the spherical coordinates
\begin{align}
ds^2 = - \left(1 - \frac{2}{r} \right) dt^2 + \left(1 - \frac{2}{r}\right)^{-1} dr^2 + r^2 d\theta^2 + r^2 d \phi^2
\end{align}
and the 4-velocity of the satellite is given by
\begin{align}
u^{\mu} = \gamma \left(1, 0, 0, \Omega\right)
\end{align}
where $\Omega = r^{-3/2} = 0.068$. For the satellite to be orbiting at a certain radius with the Schwarzschild metric, the angular velocity has to be
\begin{align}
u^{\phi} = \frac{d \phi}{d \tau} = \Omega_{\mathrm{orb}} = \frac{1}{r^{3/2}} \sqrt{\frac{r}{r - 3}} = \Omega \sqrt{\frac{r}{r - 3}} = \gamma \Omega
\end{align}
Therefore,
\begin{align}
\gamma = \sqrt{\frac{r}{r - 3}} = \sqrt{\frac{6}{6 - 3}} = \sqrt{\frac{6}{3}} = \sqrt{2} \approx 1.41
\end{align}
(ii)
\begin{align}
\frac{d J^{\nu}}{d \tau} = - \Gamma^{\nu}_{\alpha \beta} u^{\alpha} J^{\beta}
\end{align}
Here, the Christoffel symbols are given by
\begin{align}
\Gamma^{\nu}_{\alpha \beta} = \frac{1}{2} g^{\nu \sigma} \left(\partial_{\alpha} g_{\beta \sigma} + \partial_{\beta} g_{\alpha \sigma} - \partial_{\sigma} g_{\alpha \beta}\right)
\end{align}
Since the Schwarzschild metric $g_{\mu \nu}$ is diagonal in spherical coordinates, and so does $g^{\mu \nu}$, given by
\begin{align}
g_{\mu \nu} = \mathrm{diag}\left(- \left(1 - \frac{r}{2} \right), \left(1 - \frac{r}{2} \right)^{-1}, r^2, r^2 \right), \\
g^{\mu \nu} = \mathrm{diag}\left(- \left(1 - \frac{r}{2} \right)^{-1}, \left(1 - \frac{r}{2} \right), \frac{1}{r^2}, \frac{1}{r^2} \right),
\end{align}
the Christoffel symbols get simplified to
\begin{align}
\Gamma^{\nu}_{\alpha \beta} = \frac{1}{2} g^{\nu \nu} \left(\partial_{\alpha} g_{\beta \nu} + \partial_{\beta} g_{\alpha \nu} - \partial_{\nu} g_{\alpha \beta}\right)
\end{align}
Also, since $u^{\mu} = \gamma \left(1, 0, 0, \Omega\right)$, the only nonzero components of $\Gamma^{\nu}_{\alpha \beta}$ are for $\alpha = t, \phi$, so that

\begin{align}
\Gamma^{\nu}_{t \beta} &= \frac{1}{2} g^{\nu \nu} \left(\partial_{t} g_{\beta \nu} + \partial_{\beta} g_{t \nu} - \partial_{\nu} g_{t \beta}\right) = \frac{1}{2} g^{\nu \nu} \left(\partial_{\beta} g_{t \nu} - \partial_{\nu} g_{t \beta}\right)
\end{align}
$\Rightarrow$
\begin{align}
\Gamma^{t}_{t r} &= \frac{1}{2} g^{t t} \left(\partial_{r} g_{t t} - \partial_{t} g_{t r}\right) = \frac{1}{2} g^{t t} \partial_{r} g_{t t} = \frac{1}{2} \left(1 - \frac{2}{r} \right)^{-1} \frac{\partial}{\partial r} \left(1 - \frac{2}{r} \right) \nonumber\\
&= \frac{1}{2} \left(1 - \frac{2}{r} \right)^{-1} \cdot \frac{2}{r^2} = \frac{1}{r^2} \left(1 - \frac{2}{r} \right)^{-1} = \frac{1}{r \left(r - 2\right)}
\end{align}
\begin{align}
\Gamma^{r}_{t t} &= \frac{1}{2} g^{r r} \left(\partial_{t} g_{t r} - \partial_{r} g_{t t}\right) = - \frac{1}{2} g^{r r} \partial_{r} g_{t t} = - \frac{1}{2} \left(1 - \frac{2}{r} \right) \frac{\partial}{\partial r} \left(1 - \frac{2}{r} \right)\\
&= - \frac{1}{2} \left(1 - \frac{2}{r} \right) \cdot \frac{2}{r^2} = - \frac{1}{r^2} + \frac{2}{r^3}
\end{align}
Also,
\begin{align}
\Gamma^{\nu}_{\phi \beta} = \frac{1}{2} g^{\nu \nu} \left(\partial_{\phi} g_{\beta \nu} + \partial_{\beta} g_{\phi \nu} - \partial_{\nu} g_{\phi \beta}\right) = \frac{1}{2} g^{\nu \nu} \left(\partial_{\beta} g_{\phi \nu} - \partial_{\nu} g_{\phi \beta}\right)
\end{align}
$\Rightarrow$
\begin{align}
\Gamma^{\phi}_{\phi r} &= \frac{1}{2} g^{\phi \phi} \left(\partial_{r} g_{\phi \phi} - \partial_{\phi} g_{\phi r}\right) = \frac{1}{2} g^{\phi \phi} \partial_{r} g_{\phi \phi} = \frac{1}{2 r^2} \frac{\partial}{\partial r} r^2 = \frac{1}{r}
\end{align}
\begin{align}
\Gamma^{r}_{\phi \phi} &= \frac{1}{2} g^{r r} \left(\partial_{\phi} g_{\phi r} - \partial_{r} g_{\phi \phi}\right) =  - \frac{1}{2} g^{r r} \partial_{r} g_{\phi \phi} = - \frac{1}{2} \left(1 - \frac{2}{r} \right) \frac{\partial}{\partial r} r^2 = - r + 2
\end{align}
Therefore,
\begin{align}
\frac{d J^{t}}{d \tau} &= - \Gamma^{t}_{t r} u^{t} J^{r} = - \frac{\gamma}{r \left(r - 2\right)} J^{r}\\
\frac{d J^{r}}{d \tau} &= - \Gamma^{r}_{t t} u^{t} J^{t} - \Gamma^{r}_{\phi \phi} u^{\phi} J^{\phi} = - \gamma \left(\frac{1}{r^2} - \frac{2}{r^3}\right) J^{t} + \gamma \Omega \left(r - 2 \right) J^{\phi}\\
\frac{d J^{\phi}}{d \tau} &= - \Gamma^{\phi}_{\phi r} u^{\phi} J^{r} = - \frac{\gamma \Omega}{r} J^{r}
\end{align}
(iii)
\begin{align}
\frac{d}{d \tau} = \frac{dt}{d \tau} \frac{d}{d t} = u^{t} \frac{d}{d t} = \gamma \frac{d}{d t} \Leftrightarrow \frac{d}{d t} = \frac{1}{\gamma} \frac{d}{d \tau}
\end{align}
Thus,
\begin{align}
\frac{d J^{t}}{d t} &= - \frac{1}{r \left(r - 2\right)} J^{r}\\
\frac{d J^{r}}{d t} &= - \left(\frac{1}{r^2} - \frac{2}{r^3}\right) J^{t} + \Omega \left(r - 2 \right) J^{\phi}\\
\frac{d J^{\phi}}{d t} &= - \frac{\Omega}{r} J^{r}
\end{align}
(iv) From the equations of motion in the Problem.1(iii),
\begin{align}
\frac{d^2 J^{r}}{d t^2} &= - \left(\frac{1}{r^2} - \frac{2}{r^3}\right) \frac{d J^{t}}{d t} + \Omega \left(r - 2 \right) \frac{d J^{\phi}}{d t} \nonumber\\
&= + \left(\frac{1}{r^2} - \frac{2}{r^3}\right) \frac{1}{r \left(r - 2\right)} J^{r} - \Omega \left(r - 2 \right) \frac{\Omega}{r} J^{r} \nonumber\\
&= \left( \left(\frac{1}{r^2} - \frac{2}{r^3}\right) \frac{1}{r \left(r - 2\right)} - \Omega^2 \frac{r - 2}{r} \right) J^{r} = - \alpha^2 J^{r}
\end{align}
where
\begin{align}
\alpha = \sqrt{- \left( \left(\frac{1}{r^2} - \frac{2}{r^3}\right) \frac{1}{r \left(r - 2\right)} - \Omega^2 \frac{r - 2}{r} \right)} \approx \sqrt{0.0023} \approx 0.048
\end{align}
If a initial condition is given as $J^{\mu} = \left(0, J, 0, 0\right)$,
\begin{align}
\frac{d J^{t}}{d t} &= - \frac{1}{r \left(r - 2\right)} J^{r} = - \beta J\\
\frac{d J^{r}}{d t} &= 0\\
\frac{d J^{\phi}}{d t} &= - \frac{\Omega}{r} J = - \delta J
\end{align}
where
\begin{align}
\beta &= \frac{1}{6 \left(6 - 2\right)} = \sqrt{\frac{1}{24}} \approx 0.0417\\
\delta &= \frac{0.068}{6} \approx 0.0011
\end{align}
Therefore, the initial condition gives
\begin{align}
J^{\mu} = \left(0, J, 0\right), \partial_{t} J^{\mu} = \left(- \beta J,0, - \delta J \right)
\end{align}
and the solution can be given as follows:
\begin{align}
\frac{d^2 J^{r}}{d t^2} = - \alpha^2 J^{r}
\end{align}
$\Rightarrow$
\begin{align}
J^{r}(t) = J \cos\left(\alpha t\right)
\end{align}
\begin{align}
\frac{d J^{t}}{d t} &= - J \beta \cos\left(\alpha t\right) \\
\frac{d J^{\phi}}{d t} &= - J \delta \cos\left(\alpha t\right)
\end{align}
$\Rightarrow$
\begin{align}
J^{t}(t) = - \frac{J \beta}{\alpha} \sin\left(\alpha t\right)\\
J^{\phi}(t) = - \frac{J \delta}{\alpha} \sin\left(\alpha t\right)
\end{align}

\section*{Problem.2}
A torus is defined by
\begin{align}
x &= \left( R + a \cos{\theta}\right) \cos{\phi}\\
y &= \left( R + a \cos{\theta}\right) \sin{\phi}\\
z &= a \sin{\theta}
\end{align}
in 3-dimensional flat space with line element $ds^2 = dx^2 + dy^2 + dz^2$.\\
(i) The transformation on the torus gives the line element in the form of
\begin{align}
ds^2 = g_{\theta \theta} d\theta^2 + g_{\phi \phi} d\phi^2 + 2 g_{\theta \phi} d\theta d\phi
\end{align}
Here,
\begin{align}
g_{\theta \theta} &= g_{\mu \nu} \frac{\partial x^{\mu}}{\partial \theta} \frac{\partial x^{\nu}}{\partial \theta} = g_{x x} \left(\frac{\partial x}{\partial \theta}\right)^2 + g_{y y} \left(\frac{\partial y}{\partial \theta}\right)^2 + g_{z z} \left(\frac{\partial z}{\partial \theta}\right)^2\nonumber \\
&= \left(- a \sin{\theta} \cos{\phi} \right)^2 + \left(a \sin{\theta} \sin{\phi} \right)^2 + \left(- a \cos{\theta}\right)^2 = a^2
\end{align}
\begin{align}
g_{\phi \phi} &= g_{\mu \nu} \frac{\partial x^{\mu}}{\partial \phi} \frac{\partial x^{\nu}}{\partial \phi} = g_{x x} \left(\frac{\partial x}{\partial \phi}\right)^2 + g_{y y} \left(\frac{\partial y}{\partial \phi}\right)^2\\
&= \left(- \left( R + a \cos{\theta}\right) \sin{\phi} \right)^2 + \left(\left( R + a \cos{\theta}\right) \cos{\phi} \right)^2 = \left( R + a \cos{\theta}\right)^2
\end{align}
\begin{align}
g_{\theta \phi} &= g_{\mu \nu} \frac{\partial x^{\mu}}{\partial \theta} \frac{\partial x^{\nu}}{\partial \phi} = g_{x x} \frac{\partial x}{\partial \theta} \frac{\partial x}{\partial \phi} + g_{y y} \frac{\partial y}{\partial \theta} \frac{\partial y}{\partial \phi}\\
&= a \sin{\theta}\cos{\phi} \left( R + a \cos{\theta}\right)\sin{\phi} - a \sin{\theta}\sin{\phi} \left( R + a \cos{\theta}\right)\cos{\phi} = 0
\end{align}
Therefore,
\begin{align}
ds^2 = a^2 d\theta^2 + \left( R + a \cos{\theta}\right)^2 d\phi^2
\end{align}
(ii) For the 2-dimensional space of the torus, we can get by setting $z=0$ in the 2-dimensional space of the torus.
Here, the Christoffel symbols are given by
\begin{align}
\Gamma^{\nu}_{\alpha \beta} = \frac{1}{2} g^{\nu \sigma} \left(\partial_{\alpha} g_{\beta \sigma} + \partial_{\beta} g_{\alpha \sigma} - \partial_{\sigma} g_{\alpha \beta}\right)
\end{align}
Since
\begin{align}
g_{\mu \nu} = \mathrm{diag}\left(0, a^2, \left( R + a \cos{\theta}\right)^2\right),
\end{align}
$\Leftrightarrow$
\begin{align}
g^{\mu \nu} = \mathrm{diag}\left(0, \frac{1}{a^2}, \frac{1}{\left( R + a \cos{\theta}\right)^{2}}\right),
\end{align}
the non-zero Christoffel symbols have to consist of the diagonal elements of $\theta, \phi$, and the non-zero derivative is
\begin{align}
\partial_{\theta} g_{\phi \phi} = - 2 a \sin{\theta} \left( R + a \cos{\theta}\right)
\end{align}
Thus, those are
\begin{align}
\Gamma^{\phi}_{\theta \phi} = \Gamma^{\phi}_{\phi \theta} = \frac{1}{2} g^{\phi \phi} \left(\partial_{\theta} g_{\phi \phi} + \partial_{\phi} g_{\theta \phi} - \partial_{\phi} g_{\theta \phi}\right) = \frac{1}{2} g^{\phi \phi} \partial_{\theta} g_{\phi \phi} = - \frac{a \sin{\theta}}{R + a \cos{\theta}}
\end{align}
\begin{align}
\Gamma^{\theta}_{\phi \phi} = \frac{1}{2} g^{\theta \theta} \left(\partial_{\phi} g_{\phi \theta} + \partial_{\phi} g_{\phi \theta} - \partial_{\theta} g_{\phi \phi}\right) = - \frac{1}{2} g^{\theta \theta} \partial_{\theta} g_{\phi \phi} = \frac{\sin{\theta} \left( R + a \cos{\theta}\right)}{a}
\end{align}
(iii)
The Riemann tensor is given by
\begin{align}
R^{\alpha}_{~\beta \mu \nu} = \partial_{\mu} \Gamma^{\alpha}_{\nu \beta} - \partial_{\nu} \Gamma^{\alpha}_{\mu \beta} + \Gamma^{\alpha}_{\mu \gamma} \Gamma^{\gamma}_{\nu \beta} - \Gamma^{\alpha}_{\nu \gamma} \Gamma^{\gamma}_{\mu \beta}
\end{align}
Since there is only one independent component of the Riemann tensor in 2-dimensional space, first we can calculate one component, and the rest can be acquired by the symmetry and some algebra. For example, I can choose one of the non-zero components as
\begin{align}
R^{\theta}_{~\phi \theta \phi} &= \partial_{\theta} \Gamma^{\theta}_{\phi \phi} - \partial_{\phi} \Gamma^{\theta}_{\theta \phi} + \Gamma^{\theta}_{\theta \gamma} \Gamma^{\gamma}_{\phi \phi} - \Gamma^{\theta}_{\phi \gamma} \Gamma^{\gamma}_{\theta \phi} = \partial_{\theta} \Gamma^{\theta}_{\phi \phi} - \Gamma^{\theta}_{\phi \phi} \Gamma^{\phi}_{\theta \phi}\\
&= \partial_{\theta} \left(\frac{\sin{\theta} \left( R + a \cos{\theta}\right)}{a}\right) - \frac{\sin{\theta} \left( R + a \cos{\theta}\right)}{a} \cdot \left( - \frac{a \sin{\theta}}{R + a \cos{\theta}} \right) \nonumber \\
&= \frac{\cos{\theta} \left( R + a \cos{\theta}\right)}{a} - \sin^2{\theta} + \sin^2{\theta} = \frac{\cos{\theta} \left( R + a \cos{\theta}\right)}{a}
\end{align}
Hence,
\begin{align}
R_{\theta \phi \theta \phi} = g_{\theta \mu}R^{\mu}_{~\phi \theta \phi} = g_{\theta \theta}R^{\theta}_{~\phi \theta \phi}
\end{align}
and from the symmetry of the Riemann tensor,
\begin{align}
R_{\phi \theta \phi \theta} = R_{\theta \phi \theta \phi}, \quad R_{\theta \phi \phi \theta} = R_{\phi \theta \theta \phi} = R_{\theta \theta \phi \phi} = R_{\phi \phi \theta \theta} = - R_{\theta \phi \theta \phi}, \quad 
\end{align}
and other components are zero. Therefore, we can list all the non-zero components of the Riemann tensor as below:
\begin{align}
R^{\theta}_{~\phi \theta \phi} &= \frac{\cos{\theta} \left( R + a \cos{\theta}\right)}{a} \\
R^{\phi}_{~\theta \phi \theta} &= g^{\phi \phi} R_{\phi \theta \phi \theta} = g^{\phi \phi} R_{\theta \phi \theta \phi} = g^{\phi \phi} g_{\theta \theta}R^{\theta}_{~\phi \theta \phi}\\
&= \frac{a^2}{\left( R + a \cos{\theta}\right)^2} R^{\theta}_{~\phi \theta \phi} = \frac{a \cos{\theta}}{R + a \cos{\theta}}\\
R^{\phi}_{~\theta \theta \phi} &= R^{\phi}_{~\phi \theta \theta} = g^{\phi \phi} R_{\phi \theta \theta \phi} = - g^{\phi \phi} g_{\theta \theta}R^{\theta}_{~\phi \theta \phi} = - R^{\phi}_{~\theta \phi \theta} = - \frac{a \cos{\theta}}{R + a \cos{\theta}} \\
R^{\theta}_{~\phi \phi \theta} &= R^{\theta}_{~\theta \phi \phi} = g^{\theta \theta} R_{\theta \phi \phi \theta} = - g^{\theta \theta} R_{\phi \theta \phi \theta} = - g^{\theta \theta} g_{\theta \theta} R^{\theta}_{~\phi \theta \phi} \nonumber \\
&= - R^{\theta}_{~\phi \theta \phi} = - \frac{\cos{\theta} \left( R + a \cos{\theta}\right)}{a}
\end{align}
(iii)
The Ricci tensor is defined as
\begin{align}
R_{\mu \nu} = R^{\alpha}_{~\mu \alpha \nu}
\end{align}
Therefore, the non-zero components are given by
\begin{align}
R_{\theta \theta} = R^{\phi}_{~\theta \phi \theta}, \quad R_{\phi \phi} = R^{\theta}_{~\phi \theta \phi}
\end{align}
so that the Ricci scalar is given by
\begin{align}
R = R^{\mu}_{~\mu} &= R^{\theta}_{~\theta} + R^{\phi}_{~\phi} = g^{\theta \theta} R_{\theta \theta} + g^{\phi \phi} R_{\phi \phi} = g^{\theta \theta} R^{\phi}_{~\theta \phi \theta} + g^{\phi \phi} R^{\theta}_{~\phi \theta \phi}\\
&= \frac{1}{a^2} \frac{a \cos{\theta}}{R + a \cos{\theta}}  + \frac{1}{\left( R + a \cos{\theta}\right)^{2}} \frac{\cos{\theta} \left( R + a \cos{\theta}\right)}{a} \nonumber \\
&= \frac{2 \cos{\theta}}{a \left(R + a \cos{\theta}\right)}
\end{align}
\end{document}