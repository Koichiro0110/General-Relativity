\documentclass[12pt]{article}
\usepackage{fullpage}
\usepackage{graphicx}
\usepackage{hyperref}
\usepackage{bm}
\usepackage{amsmath}
\usepackage{amssymb}
\usepackage{derivative}
\usepackage{bm}
\usepackage{comment}
\usepackage{cancel}
\usepackage{xcolor}
\usepackage{float}

\renewcommand{\arraystretch}{1.5}

\begin{document}
\title{General Relativity: Homework 6}
\author{Koichiro Takahashi}
\maketitle

\section*{Problem.1}
(i)\&(ii)Attached as a code.
\begin{align}
\frac{dt}{d\tau} &= e \left(1 - \dfrac{2}{r}\right)^{-1}\\
\frac{d\phi}{d\tau} &= \frac{l_z}{r^2}\\
\frac{dr}{d\tau} &= u^{r}\\
\frac{du^{r}}{d\tau} &= -\frac{1}{r^2} + \frac{l_z^2}{r^3} - \frac{3 l_z^2}{r^4}
\end{align}
(iii)
Trajectories of nearly constant radius satisfy
\begin{align}
\frac{dr}{d\tau} &= 0\\
\frac{du^{r}}{d\tau} &= 0\\
u^{r}(0) &= \sqrt{e^2 - \left(1 - \frac{2}{r_0}\right)\left(\frac{l_z^2}{r_0^2} + 1 \right)} = 0
\end{align}
These conditions are obtained by solving the following two equations for $e$ and $l_z$ at given $r(0) = r_0$.
\begin{align}
&-\frac{1}{r_0^2} + \frac{l_z^2}{r_0^3} - \frac{3l_z^2}{r_0^4} = 0\\
&\sqrt{e^2 - \left(1 - \frac{2}{r_0}\right)\left(\frac{l_z^2}{r_0^2} + 1 \right)} = 0
\end{align}
$\Leftrightarrow$
\begin{align}
&\left(\frac{1}{r_0^3} - \frac{3}{r_0^4}\right) l_z^2 = \frac{1}{r_0^2}\\
&e^2 = \left(1 - \frac{2}{r_0}\right)\left(\frac{l_z^2}{r_0^2} + 1 \right)
\end{align}
$\Leftrightarrow$
\begin{align}
&l_z^2 = \frac{1}{r_0^2\left(\frac{1}{r_0^3} - \frac{3}{r_0^4}\right)} = \frac{1}{\left(\frac{1}{r_0} - \frac{3}{r_0^2}\right)} = \frac{r_0^2}{r_0 - 3}\\
&e^2 = \left(1 - \frac{2}{r_0}\right)\left(\frac{l_z^2}{r_0^2} + 1 \right)
\end{align}
$\Leftrightarrow$
\begin{align}
&l_z = \pm \sqrt{\frac{r_0^2}{r_0 - 3}} = 4\\
&e = \pm \sqrt{\left(1 - \frac{2}{r_0}\right)\left(\frac{1}{r_0 - 3} + 1 \right)} \approx 0.962
\end{align}
The trajectory is attached as a code.

\section*{Problem.2}
The Schwarzschild metric at at $\theta = \pi/2$ is given by
\begin{align}
g_{\mu \nu} = \mathrm{diag} \left(- \left(1 - \frac{2}{r}\right), \left(1 - \frac{2}{r}\right)^{-1}, r^2, r^2 \right)
\end{align}
$\Leftrightarrow$
\begin{align}
g^{\mu \nu} = \mathrm{diag} \left(- \left(1 - \frac{2}{r}\right)^{-1}, \left(1 - \frac{2}{r}\right), 1/r^2, 1/r^2 \right)
\end{align}
(i) Let $p_{\mu}$ of a photon in spherical coordinates at $\theta = \pi/2$, can be written, in general, as
\begin{align}
p_{\mu}(\lambda) = \left(p_{t}(\lambda), p_{r}(\lambda), 0, p_{\phi}(\lambda) \right)
\end{align}
Here, the 4-velocity of an observer on a circular orbit at $r = r_0, \theta = \pi/2$ is given by
\begin{align}
u^{\mu} = \left(u^{t}, 0, 0, u^{\phi} \right)
\end{align}
Here, since the observer is on a circular orbit at radius $r$,
\begin{align}
u^{t} = \frac{r}{r - 2} = \sqrt{\frac{r}{r - 3}}, \quad u^{\phi} = \frac{l_z}{r^2} = \frac{1}{r \sqrt{r - 3}}
\end{align}
Then, the energy of the photon is given by
\begin{align}
E_0 = - u^{\mu} p_{\mu}(0) = - p_{t} u^{t} - p_{\phi} u^{\phi}
\end{align}
Also, for a photon, we have
\begin{align}
g^{\mu \nu} p_{\mu}(0) p_{\nu}(0) = g^{tt} \left(p_{t}(0)\right)^2 + g^{rr} \left(p_{r}(0)\right)^2 + g^{\phi \phi} \left(p_{\phi}(0)\right)^2 = 0
\end{align}

Thus, the 4-momentum of a photon $p^{\mu}$, with the Schwarzschild metric, in spherical coordinates, in the reference frame of an observer on a circular orbit at radius $r_0$, $\theta = \pi/2$, is given by
\begin{align}
p^{\mu} = g^{\mu \nu} p_{\mu}
\end{align}
The code is attached after the Latex pages.\\
(ii)
Assume
\begin{align}
p_{\mu} = E_0 \left(u_t, \pm \sqrt{g_{rr}}, 0, u_{\phi} \right)
\end{align}
where
\begin{align}
u_t &= g_{tt} u^{t} = - \left(1 - \frac{2}{r}\right) \sqrt{\frac{r}{r - 3}}\\
u_{\phi} &= g_{\phi \phi} u^{\phi} = \frac{r}{r \sqrt{r - 3}}
\end{align}
\begin{align}
\frac{dt}{d\lambda} &= E \left(1 - \dfrac{2}{r}\right)^{-1}\\
\frac{d\phi}{d\lambda} &= \frac{L_z}{r^2}\\
\frac{dr}{d\lambda} &= p^{r}\\
\frac{dp^{r}}{d\lambda} &= L_z^2 \frac{r - 3}{r^4}
\end{align}
with $E = - p_t, L_z = - p_\phi$, and a initial condition
\begin{align}
p^{r}(\lambda=0) &= \pm \sqrt{\frac{E^2}{c^2} - \frac{r - 2}{r^3} L_z^2}
\end{align}
Thus, if the photon was emitted at $r = r_0$,
\begin{gather}
E = E_0 \left(1 - \frac{2}{r_0}\right) \sqrt{\frac{r_0}{r_0 - 3}}
\end{gather}
\begin{gather}
L_z = - \frac{E_0}{\sqrt{r_0 - 3}}
\end{gather}
\end{document}